\documentclass[12pt]{article}

\usepackage{amsmath}
\usepackage{amssymb}
\usepackage{enumitem}
\usepackage[margin=1in]{geometry}
\usepackage{parskip}
\usepackage{times}

\newenvironment{exercise}
  {\begin{enumerate}[left=0pt] \item[\textbf{\theexercise.}]}
  {\end{enumerate} \stepcounter{exercise}}

\newenvironment{solution}{\paragraph{Solution.}}{\hfill$\blacksquare$}

\newcommand{\N}{\mathbb{N}}
\newcommand{\Z}{\mathbb{Z}}

\begin{document}

\newcounter{exercise}
\setcounter{exercise}{1}

\begin{exercise}
  Write out all functions $f : \{1, 2, 3\} \to \{a, b\}$. How many are there?
  How many are injective? How many are surjective? How many are both?
\end{exercise}

\begin{solution}
  The possible functions are as follows:

  \begin{itemize}[noitemsep]
    \item $f(1) = a, f(2) = a, f(3) = a$
    \item $f(1) = a, f(2) = a, f(3) = b$
    \item $f(1) = a, f(2) = b, f(3) = a$
    \item $f(1) = a, f(2) = b, f(3) = b$
    \item $f(1) = b, f(2) = a, f(3) = a$
    \item $f(1) = b, f(2) = a, f(3) = b$
    \item $f(1) = b, f(2) = b, f(3) = a$
    \item $f(1) = b, f(2) = b, f(3) = b$
  \end{itemize}

  There are eight possible functions, none of which are injective, six of which
  are surjective, and none of which are both.
\end{solution}

\begin{exercise}
  Write out all functions $f : \{1, 2\} \to \{a, b, c\}$. How many are there?
  How many are injective? How many are surjective? How many are both?
\end{exercise}

\begin{solution}
  The possible functions are as follows:

  \begin{itemize}[noitemsep]
    \item $f(1) = a, f(2) = a$
    \item $f(1) = a, f(2) = b$
    \item $f(1) = a, f(2) = c$
    \item $f(1) = b, f(2) = a$
    \item $f(1) = b, f(2) = b$
    \item $f(1) = b, f(2) = c$
    \item $f(1) = c, f(2) = a$
    \item $f(1) = c, f(2) = b$
    \item $f(1) = c, f(2) = c$
  \end{itemize}

  There are nine possible functions, six of which are injective, none of which
  are surjective, and none of which are both.
\end{solution}

\begin{exercise}
  Consider the function $f : \{1, 2, 3, 4, 5\} \to \{1, 2, 3, 4\}$ given by the
  table below:

  \begin{center}
    \begin{tabular}{c||c|c|c|c|c}
      $x$    & 1 & 2 & 3 & 4 & 5 \\ \hline
      $f(x)$ & 3 & 2 & 4 & 1 & 2 \\
    \end{tabular}
  \end{center}

  \begin{enumerate}[label=(\alph*)]
    \item Is $f$ injective? Explain.
    \item Is $f$ surjective? Explain.
  \end{enumerate}
\end{exercise}

\begin{solution}
  The function $f$ is not injective. Formally, this is because there exist two
  elements in the domain, namely 2 and 5, which map to 2 in the codomain. In
  fact, there exists no injective mapping with this domain and codomain, since
  the codomain is smaller than the domain. This necessitates overlap in the
  mappings.

  Additionaly, $f$ is surjective. To see this, consider the image of $f$, which
  is the set $\{1, 2, 3, 4\}$. This set is identical to the codomain of $f$, so
  the function must be surjective.
\end{solution}

\begin{exercise}
  For each function given below, determine whether or not the function is
  injective and whether or not the function is surjective.

  \begin{enumerate}[label=(\alph*)]
    \item $f : \N \to \N$ given by $f(n) = n + 4$.
    \item $f : \Z \to \Z$ given by $f(n) = n + 4$.
    \item $f : \Z \to \Z$ given by $f(n) = 5n - 8$.
    \item $f : \Z \to \Z$ given by $f(n) =
      \begin{cases}
        n/2       & \text{if $n$ is even} \\
        (n + 1)/2 & \text{if $n$ is odd.} \\
      \end{cases}$
  \end{enumerate}
\end{exercise}

\begin{solution}
  (a) This function is injective but not surjective. Let us first show that it
  is injective. Suppose that $m, n \in \N$ such that $f(m) = f(n)$. Performing
  some algebra:

  \[
    \begin{aligned}
      f(m)  & = f(n)  \\
      m + 4 & = n + 4 \\
      m     & = n     \\
    \end{aligned}
  \]

  We have $m = n$, so $f$ is injective. To see why $f$ is not surjective,
  suppose that there exists $n \in \N$ such that $f(n) = 2$. Then $n + 4 = 2$,
  and subtracting 4 from both sides gives $n = -2$. However, $-2$ is not a
  natural number, so our original assumption was contradicted. Therefore $f$
  cannot be surjective.
\end{solution}

\end{document}
