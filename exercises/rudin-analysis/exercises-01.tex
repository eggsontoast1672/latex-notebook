\documentclass{zupan}

\usepackage[shortlabels]{enumitem}

\title{Rudin Analysis: Exercises 1}
\author{Paul Zupan}
\date{\today}

\begin{document}

\maketitle

Unless the contrary is explicitly stated, all numbers that are mentioned in
these exercises are understood to be real.

\begin{exercise}{1}
  If $r$ is rational ($r \neq 0$) and $x$ is irrational, prove that $r + x$ and
  $rx$ are irrational.
\end{exercise}

\begin{solution}
  Let $r$ be a nonzero rational number and let $x$ be a real number. We prove
  these statements by their contrapositives.

  \begin{enumerate}[(a)]
    \item Suppose that $x + r = a/b$ is a rational number where $a, b \in
      \mathbb{Z}$ with $b \neq 0$. Then we can write $x$ as the difference
      between this ratio and $r$, namely $a/b - r$. The difference of rational
      numbers is rational, so we conclude that $x$ is rational. By the
      contrapositive, if $x$ is not rational, $x + r$ cannot be rational.
    \item Suppose that $rx = a/b$ is a rational number were $a, b \in
      \mathbb{Z}$ with $b \neq 0$. Since $r$ is nonzero, we can divide both
      sides by $r$, giving that $x = a/rb$. The rationals are closed under
      division, so we conclude as before that $x$ is rational. Once again, by
      the contrapositive, if $x$ were not rational, then $rx$ would not be
      rational.
  \end{enumerate}
\end{solution}

\begin{exercise}{2}
  Prove that there is no rational number whose square is 12.
\end{exercise}

\begin{solution}
  It is sufficient to show that there exists no rational number which squares
  to 3. To show this, we repeat the argument given in Rudin for the
  irrationality of the square root of 2.

  Let $x$ be a number which squares to 3. As we have proved, $x$ must be
  irrational. Note that $(2x)^2 = 4x^2 = 12$. But a rational number times an
  irrational number is irrational, so $2x$ must be irrational. Thus, there is
  no rational number which squares to 12.
\end{solution}

\begin{exercise}{3}
  Prove Proposition 1.15.
\end{exercise}

\begin{solution}
  Throughout this problem, let $F$ be a field. Additionally, let $x$, $y$, and
  $z$ be elements of $F$ with $x$ nonzero.

  \begin{enumerate}[(a), noitemsep]
    \item Suppose $xy = zx$. Then $(1/x)xy = (1/x)xz$, so $y = z$.
    \item Suppose $xy = x$. Then $(1/x)xy = (1/x)x$, so $y = 1$.
    \item Suppose $xy = 1$. Then $(1/x)xy = 1/x$, so $y = 1/x$.
    \item We have $(1/x) \cdot x = x \cdot (1/x) = 1$, so $1/(1/x) = 1$.
  \end{enumerate}
\end{solution}

\begin{exercise}{4}
  Let $E$ be a nonempty subset of an ordered set; suppose $\alpha$ is a lower
  bound of $E$ and $\beta$ is an upper bound of $E$. Prove that $\alpha \leq
  \beta$.
\end{exercise}

\begin{solution}
  Let $x \in E$. Since $\alpha$ is a lower bound of $E$, we know that $\alpha
  \leq x$. Additionally, since $\beta$ is an upper bound of $E$, we know that
  $x \leq \beta$. By the transitive property of $\leq$, we see that $\alpha
  \leq \beta$ as required.
\end{solution}

\begin{exercise}{5}
  Let $A$ be a nonempty set of real numbers which is bounded below. Let $-A$ be
  the set of all numbers $-x$, where $x \in A$. Prove that

  \[\inf{A} = -\sup(-A).\]
\end{exercise}

\begin{solution}
  Let $\alpha = \sup(-A)$. First, we aim to show that $-\alpha$ is a lower
  bound for $A$. Second, we aim to show that $-\alpha$ is the greatest lower
  bound of $A$.

  Since $\alpha$ is an upper bound for $-A$, for all $x \in A$, we know that
  $-x \leq \alpha$. By multiplying both sides by $-1$, we can see that $-\alpha
  \leq x$. Thus, $-\alpha$ is a lower bound for $A$. Next, since $\alpha$ is
  the least upper bound of $-A$, for all $\beta < \alpha$, there exists $x \in
  A$ such that $\beta < -x$. Let $\gamma = -\beta$. Based on the previous
  assumption, for all $\gamma > -\alpha$, there exists $x \in A$ such that $x <
  \gamma$. This means precisely that $-\alpha$ is the greatest lower bound of
  $A$ as required.
\end{solution}

\begin{exercise}{6}
  Fix $b > 1$.

  \begin{enumerate}[(a), noitemsep]
    \item If $m, n, p, q$ are integers, $n > 0$, $q > 0$, and $r = m/n = p/q$,
      prove that \[(b^m)^{1 / n} = (b^p)^{1 / q}.\] Hence it makes sense to
      define $b^r = (b^m)^{1 / n}$·
    \item Prove that $b^{r + s} = b^rb^s$ if $r$ and $s$ are rational.
    \item If $x$ is real, define $B(x)$ to be the set of all numbers $b^t$,
      where $t$ is rational and $t < x$. Prove that \[b^r = \sup{B(r)}\] when
      $r$ is rational. Hence it makes sense to define \[b^x = \sup{B(x)}\] for
      every real $x$.
    \item Prove that $b^{x + y} = b^xb^y$ for all real $x$ and $y$.
  \end{enumerate}
\end{exercise}

\begin{exercise}{8}
  Prove that no order can be defined in the complex field that turns it into an
  ordered field. \textit{Hint:} $-1$ is a square.
\end{exercise}

\begin{solution}
  In particular, $i^2 = -1 < 0$, which contradicts the ordered field axioms.
  Therefore we cannot define an ordered field structure for the complex
  numbers.
\end{solution}

\begin{exercise}{9}
  Suppose $z = a + bi$, $w = c + di$. Define $z < w$ if $a < c$, and also if $a
  = c$ but $b < d$. Prove that this turns the set of all complex numbers into
  an ordered set. (This type of order relation is ca11ed a dictionary order, or
  lexicographic order, for obvious reasons.) Does this ordered set have the
  least-upper-bound property ?
\end{exercise}

\begin{solution}
  Let $z = a + bi$ and $w = c + di$ be complex numbers. We first show that
  trichotomy holds. In the case that $a < c$ or $a > c$, then we have $z < w$
  or $z > w$ respectively. If $a = c$, then $b < d$ or $b > d$ would imply that
  $z < w$ or $z > w$ respectively. If we have $a = c$ and $b = d$, then $z =
  w$. We have exhausted every case, and the ordering is always well defined, so
  trichotomy holds.

  Now let $v = e + fi$ be a complex number. Suppose that $z < w$ and $w < v$.
  We examine the possible cases:

  \begin{itemize}[noitemsep]
    \item If $a < c$ and $c < e$, then $a < e$, so $z < v$.
    \item If $a < c$ and $c = e$, then $a < e$, so $z < v$.
    \item If $a = c$ and $c < e$, then $a < e$, so $z < v$.
    \item If $a = c$ and $c = e$, then $b < d$ and $d < f$, so $b < f$ and $z < v$.
  \end{itemize}

  In every case we get that $z < v$, so transitivity holds. We conclude that
  the dictionary ordering on the complex numbers does indeed constitute a
  well-defined ordering.
\end{solution}

\begin{exercise}{10}
  Suppose $z = a + bi$, $w = u + iv$, and \[a = \left(\frac{\abs{w} + u}{2}
  \right)^{1/2}, \quad b = \left(\frac{\abs{w} - u}{2}\right)^{1/2}.\] Prove
  that $z^2 = w$ if $v \geq 0$ and that $(\bar{z})^2 = w$ if $v \leq 0$.
  Conclude that every complex number (with one exception!) has two complex
  square roots.
\end{exercise}

\begin{exercise}{11}
  If $z$ is a complex number, prove that there exists an $r \geq 0$ and a
  complex number $w$ with $\abs{w} = 1$ such that $z = rw$. Are $w$ and $r$
  always uniquely determined by $z$?
\end{exercise}

\begin{solution}
  Let $z = a + bi$ be a complex number. Choose $r = \abs{z}$ and $w =
  z/\abs{z}$. First, the norm of a complex number is always nonnegative, so $r
  \geq 0$. Next, we compute the norm of $w$:

  \[
    \begin{aligned}
      \abs{w}
        & = \sqrt{(a/\abs{z})^2 + (b/\abs{z})^2} \\
        & = \sqrt{a^2/\abs{z}^2 + b^2/\abs{z}^2} \\
        & = \sqrt{(a^2 + b^2)/\abs{z}^2} \\
        & = \sqrt{a^2 + b^2}/\abs{z} \\
        & = \abs{z}/\abs{z} \\
        & = 1 \\
    \end{aligned}
  \]

  So $w$ has length 1 as required. Finally, we observe that $rw =
  \abs{z}(z/\abs{z}) = z$. We had no choice in computing $r$ and $w$, and their
  only dependency is the initial choice of the complex number $z$. Thus, these
  two components are uniquely determined by $z$.
\end{solution}

\begin{exercise}{12}
  If $z_1, \dots, z_n$ are complex, prove that \[\abs{z_1 + z_2 + \cdots + z_n}
  \leq \abs{z_1} + \abs{z_2} + \cdots + \abs{z_n}.\]
\end{exercise}

\begin{solution}
  \[
    \begin{aligned}
      \abs{z_1 + \cdots + z_n}
        & = \sqrt{(a_1 + \cdots + a_n)^2 + (b_1 + \cdots + b_n)^2} \\
        & \geq \sqrt{a_1^2 + \cdots + a_n^2 + b_1^2 + \cdots + b_n^2} \\
    \end{aligned}
  \]
\end{solution}

\begin{exercise}{15}
  Under what conditions does equality hold in the Schwarz inequality?
\end{exercise}

\begin{exercise}{20}
  With reference to the Appendix, suppose that property (III) were omitted from
  the definition of a cut. Keep the same definitions of order and addition.
  Show that the resulting ordered set has the least-upper-bound property, that
  addition satisfies axioms (A1) to (A4) (with a slightly different
  zero-element!) but that (A5) fails.
\end{exercise}

\begin{solution}

\end{solution}

\end{document}
