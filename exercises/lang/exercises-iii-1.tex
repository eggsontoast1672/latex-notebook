\documentclass[12pt]{article}

\usepackage{amsmath}
\usepackage{amssymb}
\usepackage{enumitem}
\usepackage[margin=1in]{geometry}
\usepackage{parskip}
\usepackage{times}

\newenvironment{exercise}[1]
  {\begin{enumerate}[left=0pt] \item[\textbf{#1.}]}
  {\end{enumerate}}

\newenvironment{solution}{\paragraph{Solution.}}{\hfill$\blacksquare$}

\newcommand{\R}{\mathbb{R}}

\begin{document}

\begin{exercise}{1}
  In Example 3, give $Df$ as a function of $x$ when $f$ is the function:

  \begin{enumerate}[label=(\alph*), noitemsep]
    \item $f(x) = \sin x$
    \item $f(x) = e^x$
    \item $f(x) = \log x$
  \end{enumerate}
\end{exercise}

\begin{solution}
  For each function, we simply have to compute its derivative. Assume that $\log
  x$ here means $\log_{10} x$.

  \begin{enumerate}[label=(\alph*), noitemsep]
    \item $Df(x) = \cos x$
    \item $Df(x) = e^x$
    \item $Df(x) = \frac{1}{x\ln 10}$
  \end{enumerate}
\end{solution}

\begin{exercise}{2}
  Prove the statement about translations in Example 13.
\end{exercise}

\begin{solution}
  There are two facts which we must show:

  \begin{enumerate}
    \item If $u_1$, $u_2$ are elements of $V$, then $T_{u_1 + u_2} = T_{u_1} \circ T_{u_2}$.
    \item If $u$ is an element of $V$, then $T_u : V \to V$ has an inverse
      mapping which is nothing but the translation $T_{-u}$.
  \end{enumerate}

  First, let $u_1$ and $u_2$ be vectors in $V$. Then for any vector $v$ of $V$,
  the following holds:

  \[
    \begin{aligned}
      T_{u_1 + u_2}(v) & = v + u_1 + u_2 = v + u_2 + u_1          \\
                       & = T_{u_2}(v) + u_1 = T_{u_1}(T_{u_2}(v)) \\
                       & = (T_{u_1} \circ T_{u_2})(v)             \\
    \end{aligned}
  \]

  Next, let $u$ be a vector in $V$. For any vector $v$ in $V$, the following
  holds:

  \[
    \begin{aligned}
      (T_{-u} \circ T_u)(v) & = v + u - u = v \\
      (T_u \circ T_{-u})(v) & = v - u + u = v \\
    \end{aligned}
  \]

  So the properties hold as required.
\end{solution}

\begin{exercise}{3}
  In Example 5, give $L(X)$ when $X$ is the vector:

  \begin{enumerate}[label=(\alph*), noitemsep]
    \item $(1, 2, -3)$
    \item $(-1, 5, 0)$
    \item $(2, 1, 1)$
  \end{enumerate}
\end{exercise}

\begin{solution}
  \begin{enumerate}[label=(\alph*), noitemsep]
    \item $L(X) = 2 \cdot 1 + 3 \cdot 2 - 1 \cdot (-3) = 11$
    \item $L(X) = 2 \cdot (-1) + 3 \cdot 5 - 1 \cdot 0 = 13$
    \item $L(X) = 2 \cdot 2 + 3 \cdot 1 - 1 \cdot 1 = 6$
  \end{enumerate}
\end{solution}

\begin{exercise}{4}
  Let $F : \R \to \R^2$ be the mapping such that $F(t) = (e^t, t)$. What is
  $F(1)$, $F(0)$, $F(-1)$?
\end{exercise}

\begin{solution}
  \begin{itemize}[noitemsep]
    \item $F(1) = (e, 1)$
    \item $F(0) = (1, 0)$
    \item $F(-1) = (1/e, -1)$
  \end{itemize}
\end{solution}

\end{document}
