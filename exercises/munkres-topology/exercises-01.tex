\documentclass{zupan}

\usepackage{amsthm}
\usepackage[shortlabels]{enumitem}

\begin{document}

\begin{problem}{1}
  Check the distributive laws for $\cup$ and $\cap$ and DeMorgan's laws.
\end{problem}

\begin{solution}
  The distributive laws for set union and intersection follow from the
  distributive laws for logical operations.

  \[
    \begin{aligned}
      A \cup (B \cap C)
        & = \{x \mid x \in A \lor (x \in B \land x \in C)\} \\
        & = \{x \mid (x \in A \lor x \in B) \land (x \in A \lor x \in C)\} \\
        & = (A \cup B) \cap (A \cup C) \\
    \end{aligned}
  \]

  \[
    \begin{aligned}
      A \cap (B \cup C)
        & = \{x \mid x \in A \land (x \in B \lor x \in C)\} \\
        & = \{x \mid (x \in A \land x \in B) \lor (x \in A \land x \in C)\} \\
        & = (A \cap B) \cup (A \cap C) \\
    \end{aligned}
  \]

  Additionally, DeMorgan's laws in the context of set theory are a direct
  result of how logical negation interacts with binary logical operations. Note
  that we can duplicate any operand in a chain of AND operations with no cost.

  \[
    \begin{aligned}
      A - (B \cup C)
        & = \{x \mid x \in A \land \neg (x \in B \lor x \in C)\} \\
        & = \{x \mid x \in A \land x \notin B \land x \notin C\} \\
        & = \{x \mid x \in A \land x \notin B \land x \in A \land x \notin C\} \\
        & = (A - B) \cap (A - C) \\
    \end{aligned}
  \]

  \[
    \begin{aligned}
      A - (B \cap C)
        & = \{x \mid x \in A \land \neg (x \in B \land x \in C)\} \\
        & = \{x \mid x \in A \land (x \notin B \lor x \notin C)\} \\
        & = \{x \mid (x \in A \land x \notin B) \lor (x \in A \land x \notin C)\} \\
        & = (A - B) \cup (A - C) \\
    \end{aligned}
  \]
\end{solution}

\begin{problem}{2}
  Determine which of the following statements are true for all sets $A$, $B$,
  $C$, and $D$. If a double implication fails, determine whether one or the
  other of the possible implications holds. If an equality fails, determine
  whether the statement becomes true if the ``equals'' symbol is replaced by
  one or the other of the inclusion symbols $\subset$ or $\supset$.

  \begin{enumerate}[(a), noitemsep]
    \item $A \subset B$ and $A \subset C \iff A \subset (B \cup C)$.
    \item $A \subset B$ or $A \subset C \iff A \subset (B \cup C)$.
    \item $A \subset B$ and $A \subset C \iff A \subset (B \cap C)$.
    \item $A \subset B$ or $A \subset C \iff A \subset (B \cap C)$.
    \item $A - (A - B) = B$.
    \item $A - (B - A) = A - B$.
    \item $A \cap (B - C) = (A \cap B) - (A \cap C)$.
    \item $A \cup (B - C) = (A \cup B) - (A \cup C)$.
    \item $(A \cap B) \cup (A - B) = A$.
    \item $A \subset C$ and $B \subset D \implies (A \times B) \subset (C \times D)$.
    \item The converse of (j).
    \item The converse of (j), assuming that $A$ and $B$ are nonempty.
    \item $(A \times B) \cup (C \times D) = (A \cup C) \times (B \cup D)$.
    \item $(A \times B) \cap (C \times D) = (A \cap C) \times (B \cap D)$.
    \item $A \times (B - C) = (A \times B) - (A \times C)$.
    \item $(A - B) \times (C - D) = (A \times C - B \times C) - A \times D$.
    \item $(A \times B) - (C \times D) = (A - C) \times (B - D)$.
  \end{enumerate}
\end{problem}

\begin{problem}{3}
  \begin{enumerate}[(a), noitemsep]
    \item Write the contrapositive and converse of the following statement:
      ``If $x < 0$, then $x^2 - x > 0$,'' and determine which (if any) of the
      three statements are true.
    \item Do the same for the statement ``If $x > 0$, then $x^2 - x > 0$.''
  \end{enumerate}
\end{problem}

\begin{solution}
  (a) For the first statement, the contrapositive and converse are the
  following (in that order):

  \begin{enumerate}[noitemsep]
    \item If $x^2 - x \leq 0$, then $x \geq 0$.
    \item If $x^2 - x > 0$, then $x < 0$.
  \end{enumerate}

  The first statement is true. If $x$ is negative, then both $-x$ and $x^2$
  must be strictly positive quantities. Thus, their sum is greater than zero as
  claimed. The contrapositive of this statement follows immediately. Finally,
  the last statement is false. Consider $x = 2$. Then $x^2 - x = 2$ is greater
  than zero, but $x$ itself is also greater than zero.

  (b) The contrapositive and converse are as follows:

  \begin{enumerate}[noitemsep]
    \item If $x^2 - x \leq 0$, then $x \leq 0$.
    \item If $x^2 - x > 0$, then $x > 0$.
  \end{enumerate}

  The first statement is false with counterexample $x = 1$. Of course, the
  contrapositive is also false. The converse statement is false as well with
  counterexample $x = -2$.
\end{solution}

\begin{problem}{4}
  Let $A$ and $B$ be sets of real numbers. Write the negation of each of the
  following statements:

  \begin{enumerate}[(a), noitemsep]
    \item For every $a \in A$, it is true that $a^2 \in B$.
    \item For at least one $a \in A$, it is true that $a^2 \in B$.
    \item For every $a \in A$, it is true that $a^2 \notin B$.
    \item For at least one $a \notin A$, it is true that $a^2 \in B$.
  \end{enumerate}
\end{problem}

\begin{solution}
  \begin{enumerate}[(a), noitemsep]
    \item For at least one $a \in A$, it is false that $a^2 \in B$.
    \item For every $a \in A$, it is false that $a^2 \in B$.
    \item For at least one $a \in A$, it is false that $a^2 \notin B$.
    \item For every $a \notin A$, it is false that $a^2 \in B$.
  \end{enumerate}
\end{solution}

\begin{problem}{5}
  Let $\mathscr{A}$ be a nonempty collection of sets. Determine the truth of
  each of the following statements and of their converses:

  \begin{enumerate}[(a), noitemsep]
    \item $x \in \bigcup_{A \in \mathscr{A}} A \Rightarrow x \in A$ for at
      least one $A \in \mathscr{A}$.
    \item $x \in \bigcup_{A \in \mathscr{A}} A \Rightarrow x \in A$ for every
      $A \in \mathscr{A}$.
    \item $x \in \bigcap_{A \in \mathscr{A}} A \Rightarrow x \in A$ for at
      least one $A \in \mathscr{A}$.
    \item $x \in \bigcap_{A \in \mathscr{A}} A \Rightarrow x \in A$ for every
      $A \in \mathscr{A}$.
  \end{enumerate}
\end{problem}

\begin{solution}
  \begin{enumerate}[(a), noitemsep]
    \item Statement is true, converse is true.
    \item Statement is false, converse is true.
    \item Statement is true, converse is false.
    \item Statement is true, converse is true.
  \end{enumerate}
\end{solution}

\begin{problem}{7}
  Given sets $A$, $B$, and $C$, express each of the following sets in terms of
  $A$, $B$, and $C$, using the symbols $\cup$, $\cap$, and $-$.

  \[
    \begin{aligned}
      D & = \{x \mid x \in A \text{ and } (x \in B \text{ or } x \in C)\}, \\
      E & = \{x \mid (x \in A \text{ and } x \in B) \text{ or } x \in C\}, \\
      F & = \{x \mid x \in A \text{ and } (x \in B \implies x \in C)\}. \\
    \end{aligned}
  \]
\end{problem}

\begin{solution}
  In general, we can just change AND into intersection and OR into union.

  \[
    \begin{aligned}
      D & = A \cap (B \cup C) \\
      E & = (A \cap B) \cup C \\
      F & = A - (B - C) \\
    \end{aligned}
  \]
\end{solution}

\begin{problem}{8}
  If a set $A$ has two elements, show that $\mathscr{P}(A)$ has four elements.
  How many elements does $\mathscr{P}(A)$ have if $A$ has one element? Three
  elements? No elements? Why is $\mathscr{P}(A)$ called the power set of $A$?
\end{problem}

\begin{solution}
  Suppose that $A$ is the set $\{a, b\}$. Then the power set of $A$ is the set
  $\{\varnothing, \{a\}, \{b\}, \{a, b\}\}$. It is easy to see that this set
  has four elements. In general, if $A$ has $n$ elements, then $\mathscr{P}(A)$
  contains $2^n$ elements.

  \begin{proof}
    Let $A$ be a set of $n$ elements. A subset of $A$ can have anywhere from
    zero to $n$ elements. So choosing a subset of $A$ amounts to picking $k$
    elements from $n$, where $0 \leq k \leq n$. Thus, the number of possible
    subsets of $A$ is

    \[\sum_{k = 0}^n \binom{n}{k}.\]

    We can apply the binomial theorem to this summation to see the following:

    \[
      \begin{aligned}
        \sum_{k = 0}^n \binom{n}{k}
          & = \sum_{k = 0}^n \binom{n}{k}1^{n - k}1^k \\
          & = (1 + 1)^n \\
          & = 2^n \\
      \end{aligned}
    \]
  \end{proof}

  Thus, if $A$ has one element, three elements, or no elements, then the power
  set of $A$ has two elements, eight elements, and one element respectively.
  The relationship between the cardinality of $A$ and that of its power set
  makes sense of the nomenclature.
\end{solution}

\begin{problem}{9}
  Formulate and prove DeMorgan's laws for arbitrary unions and intersections.
\end{problem}

\begin{solution}
  I will prove the case for union, but the proof for the case for intersection
  is very similar.

  \[
    \begin{aligned}
      X - \bigcup_{A \in \mathscr{A}} A
        & = \{x \mid x \in X \land \neg (\bigvee_{A \in \mathscr{A}} x \in A)\} \\
        & = \{x \mid x \in X \land \bigwedge_{A \in \mathscr{A}} x \notin A)\} \\
        & = \{x \mid \bigwedge_{A \in \mathscr{A}} (x \in X \land x \notin A)\} \\
    \end{aligned}
  \]
\end{solution}

\begin{problem}{10}
  Let $\mathbb{R}$ denote the set of real numbers. For each of the following
  subsets of $\mathbb{R} \times \mathbb{R}$, determine whether it is equal to
  the cartesian product of two subsets of $\mathbb{R}$.

  \begin{enumerate}[(a), noitemsep]
    \item $\{(x, y) \mid x$ is an integer $\}$.
    \item $\{(x, y) \mid 0 < y \leq 1\}$.
    \item $\{(x, y) \mid y > x\}$.
    \item $\{(x, y) \mid x$ is not an integer and $y$ is an integer $\}$.
    \item $\{(x, y) \mid x^2 + y^2 < 1\}$.
  \end{enumerate}
\end{problem}

\begin{solution}
  In general, we can determine whether or not a set is the product of two
  subsets of the real numbers if the conditions are not intermingled. For
  instance, the first set is a product of subsets of $\mathbb{R}$ because there
  is no condition imposed on $y$, and the condition on $x$ only concerns $x$.

  On the contrary, the third subset cannot be written as a product of subsets
  of the real numbers because the set's condition involves both $x$ and $y$.
  The same is true for the fifth set.

  \begin{enumerate}[(a), noitemsep]
    \item $\mathbb{Z} \times \mathbb{R}$
    \item $\mathbb{R} \times (0, 1]$
    \item No such product exists
    \item $(\mathbb{R} - \mathbb{Z}) \times \mathbb{Z}$
    \item No such product exists
  \end{enumerate}
\end{solution}

\end{document}
