\documentclass[12pt]{article}

\usepackage{amsmath}
\usepackage{amssymb}
\usepackage{amsthm}
\usepackage{fancyhdr}
\usepackage[margin=1in]{geometry}

\pagestyle{fancy}
\fancyhead[l]{Paul Zupan}
\fancyhead[c]{Group Theory Problems}
\fancyhead[r]{\today}
\setlength{\headheight}{15pt}

\newtheorem{problem}{Problem}

\begin{document}

\section{Groups}

\begin{enumerate}
  \item Show that the set of automorphisms on a group \((G, *)\), denoted by
    \(\operatorname{Aut}(G)\), forms a group under function composition.

  \begin{proof}
    Let \(f\) and \(g\) be automorphisms on \(G\). For all \(x, y \in G\), we
    have \(gf(x * y) = g(f(x) * f(y)) = gf(x) * gf(y)\) since \(f\) and \(g\)
    are automorphisms. Thus, \(\operatorname{Aut}(G)\) is closed under function
    composition. In addition, function composition is known to be associative,
    so the associativity of the group operation follows.

    Let \(e(x) = x\) be the identity element in \(\operatorname{Aut}(G)\). For
    any automorphism \(f\) on \(G\), we have \(ef(x) = f(x)\) and \(fe(x) =
    f(x)\). Finally, given any automorphism \(f\), let its inverse be given by
    the inverse function \(f^{-1}\) which exists by the definition of an
    automorphism. Since \((\operatorname{Aut}(G), \circ)\) hence satisfies the
    group axioms, it is a group as required.
  \end{proof}
\end{enumerate}

\section{Commutator Subgroup}

Fix a group \(G\). For any two subgroups \(H \leq G\) and \(K \leq G\), the
group of commutators formed by them is written \([H, K]\). The underlying set
of this group can be written as \(\{hkh^{-1}k^{-1} \mid h \in H, k \in K\}\).
It consists of all the elements of the form \(hkh^{-1}k^{-1}\), which can
equivalently be written as \([h, k]\).

A special case is when \(H = G\) and \(K = G\), where the commutator subgroup
is \([G, G]\). We call this the commutator subgroup of \(G\).

\begin{enumerate}
  \item Fix a group \(G\). Show that the commutator subgroup \([G, G]\) is a
    normal subgroup of \(G\).
  \item Show that the quotient group \(G/[G, G]\) is abelian.
\end{enumerate}

\end{document}
