\documentclass{article}

\usepackage{amsmath}
\usepackage{amssymb}
\usepackage{tikz-cd}

\title{Monoids}
\author{Paul Zupan}
\date{November 2024}

\begin{document}

\maketitle

\section{Introduction}

Algebra is something that most, if not all of us get taught in school at some
point. Most of us learn algebra in terms of operations on numbers, some of
which are whole numbers and others are decimal numbers. However, algebra is
actually much more general than that. The objects that we work with in math
most certainly do not have to be numbers. This article aims to explain one of
the fundamental building blocks of abstract algebra, the monoid.

\section{Definition of a Monoid}

At a fundamental level, a monoid consists of three things. The first is some
set of objects, which do not have to be numbers. If they are numbers, they do
not have to be real numbers. They could be rational numbers, integers, or even
natural numbers. Regardless of what is inside, we will call that set $M$. This
set is sometimes called the "data" of the monoid.

Next, a monoid has some way of combining the elements of $M$. Formally, we call
this a binary operation. We will call that operation $\circ$. This circle can
represent any operation. If we wanted to be extra precise, we could even
specify the type signature of the operation, which is $\circ : M \times M \to
M$. In simpler terms, this means that $\circ$ takes in two elements from $M$
and combines them to form a new element of $M$.

The last thing that we need to create a monoid is a special element from $M$
that we call $1$. This should not be confused with the concrete number $1$.
This symbol does not necessarily denote a value of $1$, since the elements of
$M$ do not have to be numbers, as we recall. This is simply a placeholder for
what we call the identity element on $\circ$. This identity element along with
the binary operation are sometimes referred to as the "structure" of the
monoid.

We have covered the data, which is the set $M$, and we have covered the
structure, which is the binary operation $\circ$ and the identity element $1$.
There is one more thing we need to talk about to complete the definition of a
monoid, and that is the "properties." We can think of properties as constraints
that we put on the structure. The properties are the following:

\begin{itemize}
  \item For any three elements $a$, $b$, and $c$ in $M$, the equation $(a \circ
    b) \circ c = a \circ (b \circ c)$ must be satisfied. Another way of stating
    this fact is by saying that the binary operation $\circ$ is associative.
    This means that the order of the parentheses in an expression does not
    change the value of the expression.
  \item For any element $m$ in $M$, the equation $1 \circ m = m = m \circ 1$
    must be satisfied. This is the same thing as saying that $1$ cancels on the
    left and right hand side of any application of $\circ$, leaving the other
    operand unchanged. We could also say that $\circ$ is \textit{unital} with
    respect to $1$.
\end{itemize}

The data, structure, and properties listed above are all there is to a monoid.
To wrap the definition up, we could write any monoid as its three components
wrapped in a set of parentheses, like this: $(M, \circ, 1)$. That is all there
is to monoids!

\section{Motivation}

I would now like to explain briefly why we should care about monoids. It is
true that this idea may not apply directly to real life at first glance.
Monoids essentially encapsulate the idea of basic arithmetic with natural
numbers. Their power comes from their ability to translate our intuition about
arithmetic with numbers into intuition about other kinds of objects. In
addition, monoids form the basis for other kinds of abstract algebraic
structures.

It is rather difficult to explain the usefulness of monoids with words, so it
would perhaps be more fruitful to show some examples of monoids in action.

\section{Examples of Monoids}

One example of a monoid that we briefly touched on earlier was the natural
numbers, typically denoted $\mathbb{N}$. The natural numbers are essentially
positive whole numbers, the first few being $0$, $1$, $2$, \textit{ad
infinitum}. The binary operation for this monoid is addition, and the identity
element is the number $0$. Let us unpack this.

Most of us take for granted that addition is associative, mostly because we can
just \textit{feel} that it is. Say that we had three groups of things, group A,
group B, and group C. If we merge groups A and B, then merge group C, of course
the total number of things is the same as if we merged groups B and C then
merged A into that. If you would like to see a more rigorous proof of why
addition is associative over the natural numbers, see the appendix.

Intuitively, it is also rather easy to see why zero is the identity element. If
we have some things and we add zero things to that group, of course we will end
up with the same amount. Again, for a more rigorous proof of this fact, see the
appendix.

To sum up (ha ha), we have shown that $(\mathbb{N}, +, 0)$ is a monoid. Though
this is one of the most fundamental monoids, it is far from the only one.

Sticking with the natural numbers, we could also define the monoid
$(\mathbb{N}, \cdot, 1)$. This is the monoid of the natural numbers under
multiplication with $1$ as the identity element. In this case, the identity
element is literally the number $1$, unlike our placeholder value. It should be
mostly intuitive why these components form a monoid.

That wraps up the core of this explanation of monoids. However, monoids do not
stop there. It turns out that there are two more definitions of monoids that
are each quite interesting in their own right. A fair warning, understanding
these definitions fully requires a working knowledge of category theory. If you
have not studied much category theory, you may want to look up
\textbf{categories}, \textbf{functors}, and \textbf{natural transformations}
before proceeding, preferably in that order.

\section{From a Categorical Point of View}

Through the lens of category theory, a monoid is simply a category containing a
single object. This is quite an abstract definition, so let us break it down a
bit.

In our old definition, we had a set $M$ with an associative and unital binary
operation $\circ$ along with an identity element $1$. It turns out that each
component of this definition maps to one of the components of the definition of
a category! The single object of the category is simply a dummy object with no
qualities whatsoever. It may be tempting to take $M$ as the object of the
category, but that is not necessarily the case. That point of view can exist,
but we will examine it further later.

Really, the morphisms of the monoid category, which will henceforth be known as
$\mathcal{C}$, are the elements of $M$. This may seem strange, since it is
natural to think of morphisms as relationships like functions or binary
relations, but this is a perfectly valid way to define morphisms as well. The
only requirement of the morphisms is that if two share a common middle object,
they can be composed. Interestingly, since we only have a single object in this
category, all arrows share a common middle object (the only object), so they
can all be composed with each other. The effect is that the binary operation of
the monoid becomes morphism composition.

It will be useful to have a name for the single object in $\mathcal{C}$, so let
us call it $A$. Every object in every category must have a unique identity
morphism, which corresponds to the identity element in $M$. We can observe that
$\circ$ is unital over $M$ with respect to $1$ as morphism composition is
unital over $hom_\mathcal{C}(A, A)$ with respect to $1_A$.

In conclusion, all of the data, structure, and properties of an algebraic
monoid map perfectly onto a category. The advantage of looking at things this
way is that things that may not have looked like monoids before can actually be
monoids in disguise.

Let us revisit the mental model of a monoid in the category \textbf{Set}. If we
think of the single object $A$ as a set and consider all of the functions $A
\to A$, we actually have a brand new monoid. Functions are like numbers from
before, composition is like addition, and the identity map $x \mapsto x$ is
like the number $0$. You may have never even considered that monoid before, but
this categorical viewpoint makes it trivial to come up with.

\subsection{The Three Piece Suit Definition}

Okay, here we go. There is a reason that I call this the \textit{three piece
suit definition} and that is because it relies on a very, very large amount of
knowledge about category theory to fully grasp. We should now take a look at
monoidal categories.

A monoidal category, in essence, is a categorified monoid. In different words,
to get a monoidal category, you just take the algebraic definition of a monoid,
replace every set with a category, replace every function with a functor, and
replace every equation with a natural isomorphism. That is quite a lot to think
about at once, so let us again break this down.

Formally, a monoidal category $(\mathcal{C}, \otimes, I)$ is a category
$\mathcal{C}$ along with a binary functor $\otimes : \mathcal{C} \times
\mathcal{C} \to \mathcal{C}$, called the tensor product, and an identity object
$I \in \operatorname{ob}(\mathcal{C})$. So far, we have something that is very
similar to a regular algebraic monoid. The astute reader may have noticed that
we defined our data and structure, but where are the properties? The last part
of the definition includes three natural isomorphisms acting as coherence
conditions for the tensor product:

\begin{itemize}
  \item $\alpha_{A, B, C} : (A \otimes B) \otimes C \to A \otimes (B \otimes
    C)$, called the associator
  \item $\lambda_A : I \otimes A \to A$, called the left unitor
  \item $\rho_A : A \otimes I \to A$, called the right unitor
\end{itemize}

These natural isomorphisms must make the following diagrams commute. The first
diagram is the associativity diagram:

\begin{equation*}
  % https://tikzcd.yichuanshen.de/#N4Igdg9gJgpgziAXAbVABwnAlgFyxMJZABgBpiBdUkANwEMAbAVxiRAAp2BBAAgB0+EPAFt4PAEIBKfoJFiAwtIFCsouDwAiIAL6l0mXPkIoATOSq1GLNtxkq1EpbNVj28u3PUbJOvSAzYeAREACzm1PTMrIggvMqePOziHi7qbikO3j66+oFGRGQAjBaR1jG28amJyZUOik72Ylo5-gZBxshhxRFW0bEZrkkD6orDmtkWMFAA5vBEoABmAE4QwkhkIDgQSIU9UWwCjGgAFnQA+sBxzg7ipDzydxravosra4i7m9uIZpb7MYcGCdzpc7rd7mMni8QMtVutqFskABmagMOgAIxgDAACm18jEllhpsccCA9mUQIDgRcuGC7vJtGNCmdmn5Ye8UV8kGEQGjMTi8cECUSSWS-hSqacaWCxg9NM8WuzuQjvr8+VjcXkhSBCcTSeS+syro11JKQeC5VDtBRtEA
  \begin{tikzcd}
    ((A \otimes B) \otimes C) \otimes D \arrow[rr, "{\alpha_{A \otimes B, C, D}}"] \arrow[d, "{\alpha_{A, B, C} \otimes 1_D}"'] &  & (A \otimes B) \otimes (C \otimes D) \arrow[rr, "{\alpha_{A, B, C \otimes D}}"] &  & A \otimes (B \otimes (C \otimes D))                                              \\
    (A \otimes (B \otimes C)) \otimes D \arrow[rrrr, "{\alpha_{A, B \otimes C, D}}"']                                           &  &                                                                                &  & A \otimes ((B \otimes C) \otimes D) \arrow[u, "{1_A \otimes \alpha_{B, C, D}}"']
  \end{tikzcd}
\end{equation*}

And the second diagram is the identity diagram:

\begin{equation*}
  % https://tikzcd.yichuanshen.de/#N4Igdg9gJgpgziAXAbVABwnAlgFyxMJZABgBpiBdUkANwEMAbAVxiRAAoBBAAgB1eIeALbxuASQCUfAcNEAhEAF9S6TLnyEUAJnJVajFmx79BWEXG7sx00+e5yJSlSAzY8BIgEZSnvfWasiCDGMmbySnowUADm8ESgAGYAThBCSGQgOBBI3voBbPyMaAAWdAD6wJyk4tVyik6JKWmIGVlIOnmGQfxJxRBlIbainmUK1Ax0AEYwDAAKau6aIElY0cU4DSDJqTnUbYgd-l0gI4OyFvwTQpNQ5ZwRikA
  \begin{tikzcd}
    (A \otimes I) \otimes B \arrow[rr, "{\alpha_{A, I, B}}"] \arrow[rd, "\rho_A \otimes 1_B"'] &             & A \otimes (I \otimes B) \arrow[ld, "1_A \otimes \lambda_A"] \\
                                                                                               & A \otimes B &
  \end{tikzcd}
\end{equation*}

The fact that we are using natural isomorphisms instead of equations gives us a
lot of flexibility. We can create monoidal categories where the tensor product
is not really associative, but is \textit{basically} associative. For instance,
take the monoidal category $(\textbf{Set}, \times, \{*\})$. The cartesian
product is not exactly associative, but each of the two sets contains the same
information. $((a, b), c)$ is not the same object as $(a, (b, c))$, but they
contain precisely the same information, so we can define an isomorphism between
the two. In a similar manner, the object $a$ and the pair $(a, i)$ where $i$ is
fixed contain the same information.

Here is an aside about the diagrams. They confused me at first, but they are
actually extremely important. Instead of enforcing that the tensor product be
strictly associative, we only require that it be associative up to isomorphism.
That gives us quite a lot of freedom with what we can consider to be a monoidal
category, which we will see later. However, a consequence of this freedom is
that we now have the capability to define isomorphisms that do not make sense.

For instance, let us take an example in the category of groups \textbf{Grp}. In
this monoidal category, the tensor product is the cartesian product and the
identity object is the trivial group. We can define the alpha transformation as
$\alpha_{A, B, C}((a, b), c) = (-a, (-b, -c))$. Since we are in the category of
groups, $A$, $B$, and $C$ are all groups, so all elements of those groups have
inverses. Not to mention, since all inverses are unique, this transformation
does not lose any information, so it is an isomorphism. The problem arises when
we check the commutative diagrams. Let us check the identity diagram since it
is simpler:

\begin{equation*}
  \begin{split}
    (\rho_A \otimes 1_B)((a, 1), b)
      & = (\rho_A(a, 1), 1_B(b)) \\
      & = (a, b)
  \end{split}
\end{equation*}

That is fine, but when we check the other path:

\begin{equation*}
  \begin{split}
    ((1_A \otimes \lambda_A) \circ \alpha_{A, I, B})((a, 1), b)
      & = (1_A \otimes \lambda_A)(\alpha_{A, I, B}((a, 1), b)) \\
      & = (1_A \otimes \lambda_A)(-a, (1, -b)) \\
      & = (1_A(-a), \lambda_A(1, -b)) \\
      & = (-a, -b)
  \end{split}
\end{equation*}

The two paths through the diagram are not equal, so the diagram does not
commute. This is an example of how the coherence conditions set in place by the
diagrams prevent us from doing crazy things with the power of the isomorphisms.

With all of that out of the way, we can actually define a monoid. Fix a
monoidal category $(\mathcal{C}, \otimes, I)$. A monoid in $\mathcal{C}$,
denoted by $(M, \mu, \eta)$, is an object $M \in
\operatorname{ob}(\mathcal{C})$ along with two morphisms

\begin{itemize}
  \item $\mu : M \otimes M \to M$ called multiplication, and
  \item $\eta : I \to M$ called unit
\end{itemize}

such that the pentagon diagram

\begin{equation*}
  % https://tikzcd.yichuanshen.de/#N4Igdg9gJgpgziAXAbVABwnAlgFyxMJZABgBpiBdUkANwEMAbAVxiRAAoBZAAgB1eIeALbxunAJR8Bw0ZxABfUuky58hFACZyVWoxZse-QVhFxuXKcdNjxCpSAzY8BIgBZt1es1aIQh6SaydspOakRkAIw6Xvq+-lZBiiGqLijuUZ56Pn4KOjBQAObwRKAAZgBOEEJIZCA4EEgAzNQMdABGMAwACirO6iDlWAUAFjggmd5s-EJMljJmEQD6ckkgFVU11PVIEROxIPyMaMN0i8CcpGKXnPLBa5XViLt1DYhaupO+S-HzUjN360e722iHcH320yYAIeTS2rzBrQ63V6YV8gxGYz22UhuXkQA
  \begin{tikzcd}
    (M \otimes M) \otimes M \arrow[d, "\mu \otimes 1_M"'] \arrow[rr, "{\alpha_{M, M, M}}"] &  & M \otimes (M \otimes M) \arrow[rr, "1_M \otimes \mu"] &  & M \otimes M \arrow[d, "\mu"] \\
    M \otimes M \arrow[rrrr, "\mu"']                                                       &  &                                                       &  & M
  \end{tikzcd}
\end{equation*}

and the unitor diagram

\begin{equation*}
  % https://tikzcd.yichuanshen.de/#N4Igdg9gJgpgziAXAbVABwnAlgFyxMJZABgBpiBdUkANwEMAbAVxiRAEkACAHW4jwC28TgFkQAX1LpMufIRQBGclVqMWbETz6DhYydOx4CRAEzLq9Zq0QhNvfliFxO7CVJAZDcokoUrL6jZ6KjBQAObwRKAAZgBOEAJIZCA4EEhKqlZsvDA4dFoOTpwKAPp67nEJSGYpaYgZDHQARjAMAAoyRvIgsVhhABY4IBZq1iCldtqOwjl5bjHxifXUqUgAzNSNLe2d3ja9A0MjWTa8AkzzIJVLyauIGyBbrR1exvt9g8OZgSC8jQJNKB0MqXa7VFZ1B4BMa8WL9CAg8QUcRAA
  \begin{tikzcd}
    I \otimes M \arrow[r, "\eta \otimes 1_M"] \arrow[rd, "\lambda_M"'] & M \otimes M \arrow[d, "\mu"'] & M \otimes I \arrow[l, "1_M \otimes \eta"'] \arrow[ld, "\rho_M"] \\
                                                                       & M                             &
  \end{tikzcd}
\end{equation*}

commute. This is truly a lot to take in. I think that looking at an example is
the best way to under stand this definition.

Let us return to the monoidal category $(\textbf{Set}, \times, \{*\})$ from
earlier. A monoid in this category would be a set $M$ with two morphisms $\mu :
M \times M \to M$ and $\eta : \{*\} \to M$. Think about this for a second. In
the category of sets, morphisms are just functions, so $\mu$ is just a function
that combines two elements of $M$ into a third element. Wait a second, that
sounds just like the algebraic definition! It gets better though. A function
whose domain set is the singleton set is essentially just picking an element
from the codomain set. That means that $\eta$ is basically just an element of
$M$, at least it is isomorphic to an element of $M$.

You might already be able to guess the role that the commutative diagrams play
here. In category theory, we do not really get to use equals signs. Equality is
a very strict contract that category theorists often hesitate to sign. However,
one way that strict equality sneaks its way into category theory is through the
definition of a commutative diagram. What that means in this context is that
the pentagon diagram ensures that $\mu$ is an associative operation, and the
unitor diagram ensures that $\eta$ is the identity element. Let us drill in
deeper and explore this directly.

Let us take $M = \mathbb{N}$. We can define $\mu$ and $\eta$ directly:

\begin{itemize}
  \item $\mu(a, b) = a + b$
  \item $\eta(*) = 0$
\end{itemize}

We can now check to see if the diagrams commute:

\begin{equation*}
  \begin{split}
    (\mu \circ (1_M \otimes \mu) \circ \alpha_{M, M, M})((a, b), c)
      & = \mu((1_M \otimes \mu)(\alpha_{M, M, M}((a, b), c))) \\
      & = \mu((1_M \otimes \mu)(a, (b, c))) \\
      & = \mu(1_M(a), \mu(b, c)) \\
      & = \mu(a, b + c) \\
      & = a + (b + c)
  \end{split}
\end{equation*}

\begin{equation*}
  \begin{split}
    (\mu \circ (\mu \otimes 1_M))((a, b), c)
      & = \mu((\mu \otimes 1_M)((a, b), c)) \\
      & = \mu(\mu(a, b), 1_M(c)) \\
      & = \mu(a + b, c) \\
      & = (a + b) + c
  \end{split}
\end{equation*}

We can see that in order for the diagram to commute, the two paths must be
equal to each other, so $(a + b) + c$ must equal $a + (b + c)$. In this way,
the pentagon diagram ensures the associativity of $\mu$.

\section{Appendix}

\subsection{Proof of Associativity of Union}

To do.

\subsection{Proof of Associativity of Addition over $\mathbb{Z}$}

To do.

\end{document}
