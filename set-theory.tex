\documentclass{article}

\usepackage{amssymb}

\title{Set Theory}
\author{Paul Zupan}
\date{November 2024}

\begin{document}

\maketitle

\section{Introduction}

Have you ever thought about why numbers exist? Why does adding two numbers
together make a new number? What even is addition?

These questions may not seem relevant. Of course numbers exist. After all, I
know without a doubt that I have 10 fingers, two arms, and five coffee cups on
my desk. Neither you nor I need math to explain that. However, in the very
precise world of mathematics, it's not enough to just believe that something is
true.

It is possible to think of math as a bunch of facts floating around in space,
each tangentially related to numbers in some way. After all, what is math
without numbers? The truth is, math is much deeper than that. There are really
two kinds of facts:

\begin{enumerate}
  \item Axioms. These are things that we assume to be true. Axioms are like the
    foundation of a house. You can't build anything on top of there's nothing
    to support it.
  \item Theorems. These are facts which follow from axioms. Essentially, one
    can combine and manipulate the axioms to produce new facts.
\end{enumerate}

The thing about axioms that is kind of confusing is that axioms can be
anything. Anything at all. Of course some foundations work better for
construction than others, so not just any old axioms will be able to support
the rest of math on top of them.

\section{The ZFC Axioms}

\subsection{Axiom of Extensionality}

Any two sets which contain the same elements are themselves the same. Another
way to think about this is that the equality of two sets is defined entirely by
what is inside, rather than the manner in which they were generated. For
instance, if $A = \{1, 2, 3\}$ and $B = \{x \in \mathbb{N} : x \leq 3\}$, we
call them the same because they each contain $1$, $2$, and $3$, despite the
fact that their definitions look different. Formally, this axiom is expressed
as

$$
\forall x. \forall y. [\forall z. (z \in x \Leftrightarrow z \in y)
\Leftrightarrow x = y].
$$

In plain English, pick any two sets $x$ and $y$. Saying that $x = y$ is the
same thing as saying that if you pick a third set $z$, then $z$ is either in
both $x$ and $y$ or it is in neither. The bracketing simply act as scopes for
$x$, $y$, and $z$.

\subsection{Axiom of Regularity}

Any nonempty set $x$ contains an element $y$ such that $x$ and $y$ are
disjoint, meaning that $x$ and $y$ share no elements. The purpose of this axiom
is to avoid Russel's paradox, which is the idea that bad things happen if you
allow a set to be an element of itself. Formally,

$$
\forall x. x \neq \varnothing \Rightarrow \exists y \in x : y \cap x =
\varnothing.
$$

Note that this axiom is only part the way we ensure that Russel's paradox can't
happen. All that the axiom of regularity says is that if you pick any set $x$
which contains at least one element, then at least one element of $x$, itself
contains none of the elements that $x$ does. This doesn't stop another set, say
$z$, from being an element of $x$ such that $z \cap x \neq \varnothing$.

\subsection{Axiom Schema of Specification}

Points of confusion:

\begin{itemize}
  \item What is $\varnothing$?
  \item What is a set?
\end{itemize}

\end{document}
