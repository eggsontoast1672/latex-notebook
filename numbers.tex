\documentclass[12pt]{article}

\usepackage{amsmath}
\usepackage{amssymb}
\usepackage{enumitem}
\usepackage[margin=1in]{geometry}
\usepackage{parskip}

\newtheorem{definition}{Definition}

\begin{document}

In most math, we depend on the axioms of set theory. That means that everything
in math can be defined in terms of sets, including numbers. But how exactly are
numbers constructed from sets?

The most fundamental concept in math is the set. Everything in mathematics can
be defined in terms of sets. So, what exactly is a set?

\begin{definition}
  A set is a collection of objects, called its elements.
\end{definition}

We typically name sets with uppercase letters such as \(A\) or \(B\), whereas
elements are named by lowercase letters like \(x\) or \(y\). If a set \(A\)
contains an element \(x\), we write \(x \in A\). This point of view gives rise
to some important properties of sets.

Firstly, sets are unordered. If a set \(A\) contains two elements \(x\) and
\(y\), it does not matter in which order they are contained. We only concern
ourselves with the fact that \(A\) contains both. Secondly, we do not consider
sets to have duplicate elements. For some element \(x\), if it is contained in
\(A\), then we do not consider the situation further.

Another concept which is quite useful is that of a subset. We say that the set
\(A_0\) is a subset of \(A\) if and only if for every element \(x\) of \(A_0\),
it is true that \(x \in A\). If this holds, we write \(A_0 \subset A\). Note
that this does not exclude the possibility that \(A_0\) and \(A\) denote the
same set. If it is also true that \(A \subset A_0\), then we say that \(A_0 =
A\), for they contain precisely the same elements.

There is one set which deserves special consideration. This set is the empty
set, written \(\varnothing\). This set has the unique property that for any
element \(x\), the set \(\varnothing\) does not contain \(x\). Particularly,
this means that the empty set is a subset of every other set. This is because
the subset condition is vacuously satisfied.

Finally, there is an important operation we can perform on sets called union.
The set \(A \cup B\) contains every element which is contained in either \(A\)
or \(B\), or possibly both. For example, if \(A = \{1, 2, 3\}\) and \(B = \{3,
4\}\), then \(A \cup B = \{1, 2, 3, 4\}\).

With this information in hand, we are ready to construct the most basic set of
numbers, the natural numbers. Each number is defined in an iterative fashion as
follows:

\begin{itemize}[noitemsep]
  \item \(0 = \varnothing\)
  \item \(1 = \{0\}\)
  \item \(2 = \{0, 1\}\)
\end{itemize}

And so on. Call the set we just defined \(\mathbb{N}\). Now that we have a
definition for the natural numbers, a natural next step is to endow this set
with some structure. The first piece of structure we can define is an operation
called addition, a way to combine any two natural numbers to produce another.
First, we define an operation called succession. Define the function \(S :
\mathbb{N} \to \mathbb{N}\) via \(S(n) = n \cup \{n\}\).

With succession in hand, we can define addition recursively. Let \(+ :
\mathbb{N} \to \mathbb{N}\) be a binary operation defined with \(n + 0 = n\)
and \(m + S(n) = S(m + n)\). Interestingly, we now have all the scaffolding
required to prove that \(1 + 1 = 2\):

\begin{equation*}
  1 + 1 = 1 + S(0) = S(1 + 0) = S(1) = 2
\end{equation*}

For those curious, the proof of the fact that \(1 + 1 = 2\) is no more
complicated than this. We can also define multiplication in terms of addition.
Let \(\cdot : \mathbb{N} \to \mathbb{N}\) be defined by \(n \cdot 0 = 0\) and
\(m \cdot S(n) = m + (m \cdot n)\).

\end{document}
