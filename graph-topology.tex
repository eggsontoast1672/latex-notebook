\documentclass{amsart}

\usepackage{amssymb}
\usepackage[a4paper, margin=1in]{geometry}
\usepackage{tikz}

\usetikzlibrary{graphs}

\newtheorem{definition}{Definition}[section]

\renewcommand{\Im}{\operatorname{Im}}

\title{Topological Representations and Drawings in Graph Theory}
\author{
  \small{Paul Zupan} \\
  \footnotesize{University of Oregon} \\
  \footnotesize{email: pzupan@uoregon.edu}
}
\date{}

\begin{document}

\begin{abstract}
  While graphs are an inherently algebraic construct, it is useful to interpret
  them as topological spaces. This perspective allows us to more easily study
  concepts like connectedness and planarity using the language and tools of
  topology.
\end{abstract}

\maketitle

\section{Graphs}

\begin{definition}
  A graph \(G\) is an ordered triple \((V, E, \varphi)\) consisting of a set
  \(V\) of \textbf{vertices}, a set \(E\) of \textbf{edges}, and an
  \textbf{adjacency map} \(\varphi : E \to \mathcal{P}(V)\) such that
  \(\varphi(e)\) contains exactly two distinct vertices for all \(e \in E\).
\end{definition}

The definition that we give is for a \textbf{finite, simple} graph. By finite,
we mean that both \(V\) and \(E\) are finite sets. By simple, we mean that no
edge joins a vertex to itself, and that no pair of vertices is joined by two
distinct edges. Neither of these assumptions are made out of necessity, but for
the sake of simplicity.

Since the graphs that we will consider do not have parallel edges, any edge can
be uniquely identified by the vertices that it joins. For instance, the edge
joining vertices \(v_1\) and \(v_2\) is often written \(v_1v_2\), or \(v_2v_1\)
equivalently. This allows us to think of the incidence map implicitly.

A purely algebraic representation for graphs does not lend itself to easy
study. For this reason, we frequently draw pictures of graphs to see them more
clearly. Suppose we had a graph \(G\) with vertex set \(V = \{v_1, v_2, v_3\}\)
and edge set \(E = \{v_1v_2, v_2v_3\}\). We could draw this graph in the
following way:

\vspace{0.25in}
\begin{center}
  \tikz\graph[nodes={draw, circle}, typeset=\(v_{\tikzgraphnodetext}\)] {
    1 -- 2 -- 3
  };
\end{center}
\vspace{0.25in}

We will frequently omit the names of nodes if they are unimportant in that
particular context.

% \begin{definition}
%   Let \(G = (V, E, \varphi)\) be a graph. We say that \(u, v \in V\) are
%   \textbf{adjacent} in \(G\) if \(\varphi(e) = \{u, v\}\) for some edge \(e \in
%   E\).
% \end{definition}

\section{Topological Representations}

\begin{definition}
  Let \(G = (V, E, \varphi)\) be a graph. The \textbf{topological
  representation} of \(G\), denoted \(T(G)\), is the set

  \begin{equation*}
    V \sqcup \bigsqcup_{e \in E} [0, 1]
  \end{equation*}

  bearing the subspace topology on \(\mathbb{R}^2\).
\end{definition}

\section{Drawings}

\begin{definition}
  Let \(G = (V, E, \varphi)\) be a graph. A \textbf{drawing} of \(G\) consists
  of the following:
  
  \begin{itemize}
    \item A vertex embedding \(f : V \hookrightarrow \mathbb{R}^2\)
    \item A collection of continuous edge maps \(\{f_e : [0, 1] \to
      \mathbb{R}^2 \setminus \Im(f)\}_{e \in E}\)
  \end{itemize}

  subject to the condition that if \(\varphi(e) = \{v_1, v_2\}\), then \(f_e(0)
  = f(v_1)\) and \(f_e(1) = f(v_2)\).
\end{definition}

This definition aims to reflect the process by which one would draw a graph on
a piece of paper. One might first draw the vertices, putting one point per
vertex such that none are in the same spot. Then, the vertices would be linked
together in a clear manner such that the endpoints of the edges are attached to
the correct vertices.

\begin{definition}
  Let \(G\) be a graph. We say that \(G\) is \textbf{planar} if there exists a
  drawing \((f, \{f_e\}_{e \in E})\) of \(G\) such that for every pair of edges
  \((e_1, e_2)\) in \(G\).
\end{definition}

\end{document}
