\documentclass[12pt]{article}

\usepackage{amssymb}
\usepackage{amsthm}
\usepackage[margin=1in]{geometry}
\usepackage{parskip}

\newtheorem{lemma}{Lemma}
\newtheorem{theorem}{Theorem}

\begin{document}

\begin{lemma}
  Every nonempty subset of the integers which is bounded below has a minimal
  element.
\end{lemma}

\begin{proof}
  Let \(A\) be a nonempty subset of the integers which is bounded below. We aim
  to show that \(A\) contains a minimal element. Since \(A\) is bounded below,
  there exists an integer \(p\) such that \(p \leq a\) for all \(a \in A\).
  There are two cases:

  \begin{enumerate}
    \item Suppose that \(p \geq 1\). By transitivity, \(a \geq 1\) for all \(a
      \in A\), so \(A\) is a subset of the positive integers. By the well
      ordering principle, \(A\) must contain a minimal element.

    \item Suppose that \(p < 1\). Consider the set \(B = \{a - p + 1 \mid a \in
      A\}\). It can be verified that 1 is a lower bound for \(B\), so it
      contains only positive integers. Then by the well ordering principle,
      \(B\) contains a minimal element \(b_0\).

      Let \(a_0 = b_0 + p - 1\). We aim to show that \(a_0\) is the minimal
      element of \(A\). First, we need to show that \(a_0 \in A\). By
      construction, there exists \(a \in A\) such that \(b_0 = a - p + 1\).
      Substituting this in the equation for \(a_0\) gives \(a_0 = a\), so
      \(a_0\) is an element of \(A\).

      Next, we need to show that \(a_0\) is a lower bound for \(A\). Suppose
      that there exists \(a \in A\) such that \(a < a_0\). Then \(a < b_0 + p -
      1\), so \(a - p + 1 < b_0\). But the left hand side is an element of
      \(B\), which is a contradiction as \(b_0\) was chosen as the minimal
      element thereof. We conclude that \(a_0\) is a lower bound of \(A\) as
      required.

      Since \(a_0\) is both an element of \(A\) and a lower bound of \(A\), it
      must be the minimal element in \(A\).
  \end{enumerate}

  In both cases we see that \(A\) has a minimal element, so the proof is
  complete.
\end{proof}

\begin{theorem}
  The integers have the greatest lower bound property.
\end{theorem}

\begin{proof}
  Let \(A\) be a nonempty subset of the integers which is bounded below. By
  Theorem 1, \(A\) has a minimal element \(a_0\). It is trivial that no element
  greater than \(a_0\) can be a lower bound for \(A\), so \(a_0 = \inf{A}\).
\end{proof}

\end{document}
