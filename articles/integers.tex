\documentclass[12pt, titlepage]{article}

\usepackage{amssymb}
\usepackage[margin=1in]{geometry}

\title{Products, Relations, and the Integers}
\author{Paul Zupan}
\date{Not Published}

\newtheorem{definition}{Definition}

\begin{document}

\maketitle

\section{Introduction}

As we have seen, the foundation of mathematics lies in set theory. That is, all
mathematical objects are constructed in terms of sets.

\section{The Cartesian Product}

\begin{definition}
  Let \(A\) and \(B\) be sets. The Cartesian product of \(A\) and \(B\),
  denoted \(A \times B\), is the set of all pairs of the form \((a, b)\) where
  \(a \in A\) and \(b \in B\).
\end{definition}

\section{The Integers}

We now have the tools required to construct the integers. Integers are whole
numbers, which can be positive or negative. The first step towards this
definition is defining an equivalence relation \(\sim\) on the set \(\mathbb{N}
\times \mathbb{N}\). We say that \((a, b) \sim (c, d)\) if and only if \(a + d
= b + c\). Finally, we define the set of integers as \(\mathbb{Z} = (\mathbb{N}
\times \mathbb{N})/\sim\).

\end{document}
