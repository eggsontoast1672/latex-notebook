\documentclass[12pt, titlepage]{article}

\usepackage{amsmath}
\usepackage{amssymb}
\usepackage{enumitem}
\usepackage[margin=1in]{geometry}
\usepackage{parskip}

\title{Where Do Numbers Come From?}
\author{Paul Zupan}
\date{\today}

\newtheorem{definition}{Definition}

\begin{document}

\maketitle
\tableofcontents
\newpage

\section{Introduction}

% Why do numbers need definitions?

Numbers are a concept of infinite use in math. They are used extensively
throughout many different fields, and almost every person uses them on a daily
basis. However, dispite popular belief, numbers are not axiomatic in nature.
That is, all of the number with which we are used to working are constructed
from more basic concepts.

The common conception is that numbers are the most fundamental objects in
mathematics, but in reality, that role is served by the set. Sets are
collections of objects which have only a few simple properties, whicm make them
a great building block from which to construct further notions.

In most math, we depend on the axioms of set theory. That means that everything
in math can be defined in terms of sets, including numbers. But how exactly are
numbers constructed from sets?

The most fundamental concept in math is the set. Everything in mathematics can
be defined in terms of sets. So, what exactly is a set?

\section{Basic Set Theory}

\subsection{Definition and Properties}

Sets serve as a ground level concept in math. This means that we do not define
sets in terms of other concepts, rather we assume that a set is something which
exists and go from there. Additionally, we list the properties that sets ought
to have, called axioms. These are the foundations provided by the ZFC axioms,
or the Zermelo-Fraenkel axioms with the axiom of choice.

\begin{definition}
  A set is a collection of objects, called its elements.
\end{definition}

This is the most fundamental definition in all of math. Sets are typically
denoted by uppercase letters, whereas elements of sets are represented by
lowercase letters. If \(A\) is a set, then we express that \(a\) is an element
of that set by writing \(a \in A\). An important property of sets is that for
any given element \(a\) and any set \(A\), there exists a dichotomy by which
\(a\) is either an element of \(A\), or it is not an element of \(A\). It is
not possible for neither to be true, nor for both to be true.

In particular, sets are not necessarily ordered. If a set \(A\) contains two
elements \(x\) and \(y\), it does not matter in which order they are contained.
We only concern ourselves with the fact that \(A\) contains both. However, it
is sometimes useful to impose an ordering on a set, which we will see.
Secondly, we do not consider sets to have duplicate elements. For some element
\(x\), if it is contained in \(A\), then we do not consider the situation
further.

\subsection{Special Sets}

There are a few special sets which are of particular interest. One such type of
set is a subset. We say that the set \(A_0\) is a subset of \(A\) if and only
if for every element \(x\) of \(A_0\), it is true that \(x \in A\). If this
holds, we write \(A_0 \subset A\). Note that this does not exclude the
possibility that \(A_0\) and \(A\) denote the same set. If it is also true that
\(A \subset A_0\), then we say that \(A_0 = A\), for they contain precisely the
same elements. If \(A_0\) is a subset of \(A\) but the two sets are not equal,
we say that \(A_0\) is a \textbf{strict subset} of \(A\).

While most sets contain at least one element, we give special consideration to
the set which contains no elements, called the empty set. We denote this set by
the symbol \(\varnothing\). A fact of note regarding the empty set is that it
is always a subset of any other set. The reason why may not be obvious at
first, so we turn to the definition. Suppose \(A\) is some set. We need to show
that for every element \(a\) in the empty set, \(a\) is also an element of
\(A\). But there are no elements in the empty set, so we say that the condition
is vacuously satisfied.

\subsection{Set Operations}

There are a variety of ways that we can use sets to build new sets. This is not
an exhaustive list, but all the concepts needed to understand the rest of this
text will be covered.

Given two sets, we can join them to create a set containing all elements from
either set. This operation is called \textbf{union}. If \(A\) and \(B\) are
sets, then we write their union \(A \cup B\). An element of \(x\) is a member
of \(A \cup B\) if it is either a member of \(A\) or of \(B\), or both. For
example, if \(A = \{a, b, c\}\) and \(B = \{c, d, e\}\), then \[A \cup B = \{a,
b, c, d, e\}.\] As a side note, the existance of a set formed from the union of
two sets is guaranteed by the axiom of union from ZFC.

Next, it is useful to be able to take two elements and create an ordered pair
out of them. Suppose that \(a\) and \(b\) are elements. Then we define the pair
with \(a\) as its first element and \(b\) as its second element to be the set
\(\{\{a\}, \{a, b\}\}\). This set is often written a more condensed form,
namely \((a, b)\). Finally, we define the Cartesian product of the sets \(A\)
and \(B\) to be \[\{(a, b) \mid a \in A, b \in B\}.\] This set consists of all
pairs whose first element is from \(A\) and second element is from \(B\). The
product of \(A\) with itself is frequently shortened to \(A^2\).

There is one more construction that will come up while constructing numbers
from sets. In order to understand it, however, we first need to explore
equivalence relations.

\subsection{Equivalence Relations}

With this information in hand, we are ready to construct the most basic set of
numbers, the natural numbers. Each number is defined in an iterative fashion as
follows:

\begin{itemize}[noitemsep]
  \item \(0 = \varnothing\)
  \item \(1 = \{0\}\)
  \item \(2 = \{0, 1\}\)
\end{itemize}

And so on. Call the set we just defined \(\mathbb{N}\). Now that we have a
definition for the natural numbers, a natural next step is to endow this set
with some structure. The first piece of structure we can define is an operation
called addition, a way to combine any two natural numbers to produce another.
First, we define an operation called succession. Define the function \(S :
\mathbb{N} \to \mathbb{N}\) via \(S(n) = n \cup \{n\}\).

With succession in hand, we can define addition recursively. Let \(+ :
\mathbb{N} \to \mathbb{N}\) be a binary operation defined with \(n + 0 = n\)
and \(m + S(n) = S(m + n)\). Interestingly, we now have all the scaffolding
required to prove that \(1 + 1 = 2\):

\begin{equation*}
  1 + 1 = 1 + S(0) = S(1 + 0) = S(1) = 2
\end{equation*}

For those curious, the proof of the fact that \(1 + 1 = 2\) is no more
complicated than this. We can also define multiplication in terms of addition.
Let \(\cdot : \mathbb{N} \to \mathbb{N}\) be defined by \(n \cdot 0 = 0\) and
\(m \cdot S(n) = m + (m \cdot n)\).

\end{document}
