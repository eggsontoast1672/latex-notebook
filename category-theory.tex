\documentclass{article}

\usepackage{amssymb}
\usepackage{tikz-cd}

\begin{document}

\section{Classical Category Theory}

\subsection{Universal Properties}

\paragraph{What are initial and terminal objects in a category?}

A terminal object is a special object in a category which is "final" in a
universal way. Formally, an object \(T \in \mathcal{C}\) is terminal if and
only if for every object \(X \in \mathcal{C}\) there exists a unique arrow \(X
\to T\). Not all categories have terminal objects, but most of the categories
that are commonly studied do. It also happens that all terminal objects in a
fixed category are uniquely isomorphic.

\textit{Proof} \: Let \(T_1, T_2 \in \mathcal{C}\) be terminal objects. Since
\(T_1\) is terminal, there exists a unique morphism \(f : T_2 \to T_1\).
Similarly, there exists a unique morphism \(g : T_1 \to T_2\). Their
composition is an arrow \(g \circ f : T_1 \to T_1\). By the universal property
of \(T_1\), there is a unique arrow \(T_1 \to T_1\), so \(g \circ f =
1_{T_1}\). Thus, \(f\) and \(g\) are a unique pair of inverses, so \(T_1\) is
uniquely isomorphic to \(T_2\) as required. \(\square\)

In the category \textbf{Set}, the terminal object is the singleton set, denoted
by \(1 = \{*\}\). Think about any arrow \(A \to 1\) from an arbitrary set \(A\)
to the singleton set. The only function that we could possibly define between
those two sets is the constant function \(a \mapsto *\). We simply have no
choice as to where to send \(a\).

In a preorder category \(\mathcal{P}\), the terminal object follows the
intuitive idea of the "largest" element. For example's sake, take the preorder
\((\{1, 2, 3\}, \leq)\). If we think of this preorder as a category, the
terminal object would be \(3\) because for any element \(x\), we have \(x \leq
3\). We don't have to worry about uniqueness of arrows here since every arrow
in a preorder category is unique.

The dual concept of a terminal object is an initial object, and it encapsulates
the concept of something which comes "first" in a category. An object \(I \in
\mathcal{C}\) is initial if and only if for every object \(X \in \mathcal{C}\)
there exists a unique arrow \(I \to X\). We could also say that an object is
initial in \(\mathcal{C}\) if and only if it is terminal in
\(\mathcal{C}^{op}\). This duality induces some immediate facts about initial
objects, the most prominent of which being that any two terminal objects must
be uniquely isomorphic. Indeed, not all categories have initial objects.

In \textbf{Set}, the initial object is the empty set \(\varnothing\).
Considering any arrow \(\varnothing \to A\), the only possible function which
fits is the empty function. We need to assign an output value to every input
value, but since there are no input values to assign to, we are already done
and had no choice.

In a preorder category, the initial object is simply the smallest object in the
set, for similar reasons as the terminal object.

\paragraph{What are cones and cocones?}

A cone over a diagram is hard to explain, but easy to picture. Consider any
diagram of a category. For the sake of example, I will use a category which has
only two objects \(A\) and \(B\), with no nontrivial morphisms. To construct a
cone over this diagram, add a vertex object \(V\) and a morphism from \(V\) to
each object in the original diagram. We now have the following diagram:

\[
  \begin{tikzcd}
      & V \arrow[ld] \arrow[rd] &   \\
    A &                         & B
  \end{tikzcd}
\]

Here is another example. Suppose we start with a cospan structure consisting of
an object \(C\) and two morphisms pointing into \(C\) from objects \(A\) and
\(B\):

\[
  \begin{tikzcd}
    A \arrow[r] & C & B \arrow[l]
  \end{tikzcd}
\]

Since we have an arrow in the starting diagram this time, it is important to
note that a well-defined cone over a diagram must result in a commuting
diagram. The cone would then look like this:

\[
  \begin{tikzcd}
                & V \arrow[ld] \arrow[rd] \arrow[d, dashed] &             \\
    A \arrow[r] & C                                         & B \arrow[l]
  \end{tikzcd}
\]

The arrow down the center is dashed because it can be inferred from the others
by composition. Quickly, a cocone is the dual notion of a cone. Instead of a
vertex \(V\) and arrows pointing out of it to the other objects, the cocone has
a vertex \(V\) with arrows pointing from the other objects \textit{into} it. A
cocone over the previous example would look like this:

\[
  \begin{tikzcd}
    A \arrow[r] \arrow[rd] & C \arrow[d, dashed] & B \arrow[l] \arrow[ld] \\
                           & V                   &
  \end{tikzcd}
\]

\paragraph{What are limits and colimits?}

The limit over a diagram \(D\) is the terminal object in the category of cones
over \(D\). Dually, the colimit of a diagram \(D\) is the initial object in the
category of cocones over \(D\). Let us look at limits first.

We should first define what the category of cones over \(D\) means. Consider
the category from before which contains only the objects \(A\) and \(B\).
Henceforth, call the diagram which illustrates this category \(D\). In the
category of cones over \(D\), an object is simply the following cone:

\[
  \begin{tikzcd}
      & V \arrow[ld] \arrow[rd] &   \\
    A &                         & B
  \end{tikzcd}
\]

The objects in this category, then, are simply ordinary objects with the extra
requirement that they must have morphisms going to \(A\) and \(B\). A morphism
in this category would look like this:

\[
  \begin{tikzcd}
      & U \arrow[ldd, bend right] \arrow[rdd, bend left] \arrow[d] &   \\
      & V \arrow[ld] \arrow[rd]                                    &   \\
    A &                                                            & B
  \end{tikzcd}
\]

The morphism in question here is the one \(U \to V\). Note that \(U\) and \(V\)
are both the vertices of cones over \(D\). We are now ready for the idea of a
limit. The limit over the diagram \(D\) is a cone \(L\) over \(D\) such that
for any other cone \(X\) over \(D\), there exists a unique morphism \(X \to L\)
making the entire diagram commute. Here is the situation illustrated:

\[
  \begin{tikzcd}
      & X \arrow[ldd, bend right] \arrow[rdd, bend left] \arrow[d, dashed] &   \\
      & L \arrow[ld] \arrow[rd]                                            &   \\
    A &                                                                    & B
  \end{tikzcd}
\]

If you have studied a lot of category theory before, you may have noticed that
this diagram exactly describes \(L\) as the product of \(A\) and \(B\), and you
would be right to think that. The product \(A \times B\) can simply be
described as the limit over \(A\) and \(B\). In a similar fashion, the
coproduct \(A + B\) is just the colimit over the objects \(A\) and \(B\).

Taking the limit of other diagrams also leads to interesting constructions. Let
us return to another example introduced before, the diagram of the cospan \(A
\rightarrow C \leftarrow B\). If we take the limit of this diagram, we get the
following result:

\[
  \begin{tikzcd}
    X \arrow[rd, dashed] \arrow[rrd, bend left] \arrow[rdd, bend right] &                       &             \\
                                                                        & L \arrow[d] \arrow[r] & B \arrow[d] \\
                                                                        & A \arrow[r]           & C
  \end{tikzcd}
\]

This is precisely equivalent to another categorical construction, the pullback.
Again, if we take the colimit of the \textit{span} \(A \leftarrow C \rightarrow
B\), we get a diagram equivalent to that of a pushout. Limits and colimits are
very interesting and useful universal properties which give rise to all sorts
of cool constructions.

\end{document}
