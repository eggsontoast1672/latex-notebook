\documentclass{article}

\usepackage{amsmath}
\usepackage{amssymb}

\begin{document}

Assume that every function is defined as \(\mathbb{R} \to \mathbb{R}\) and
every variable is a real number, unless specified otherwise. Round each answer
to three decimal places unless otherwise specified.

\section{Limits and Continuity}

\subsection{Road Construction}

A car's position \(s(t)\) on a highway is given by \(s(t) = 3t^2 + 2t - 5\) (in
meters), where \(t\) is in seconds. Find the instantaneous velocity of the car
at \(t = 2\) seconds.

\textit{Solution} \: We need to find the rate of change of \(s\) at \(t = 2\).
We can use a limit expression to achieve this:

\[
  \begin{split}
    \lim_{t \rightarrow 2} \frac{s(t) - s(2)}{t - 2}
      & = \lim_{t \rightarrow 2} \frac{3t^2 + 2t - 5 - 3 \cdot 2^2 - 2 \cdot 2 + 5}{t - 2} \\
      & = \lim_{t \rightarrow 2} \frac{3t^2 + 2t - 3 \cdot 4 - 4}{t - 2} \\
      & = \lim_{t \rightarrow 2} \frac{3t^2 + 2t - 12 - 4}{t - 2} \\
      & = \lim_{t \rightarrow 2} \frac{3t^2 + 2t - 16}{t - 2} \\
      & = \lim_{t \rightarrow 2} \frac{3t^2 + 8t - 6t - 16}{t - 2} \\
      & = \lim_{t \rightarrow 2} \frac{t(3t + 8) - 2(3t + 8)}{t - 2} \\
      & = \lim_{t \rightarrow 2} \frac{(3t + 8)(t - 2)}{t - 2} \\
      & = \lim_{t \rightarrow 2} (3t + 8) \\
      & = 3 \cdot 2 + 8 \\
      & = 6 + 8 \\
      & = 14
  \end{split}
\]

Thus, the instantaneous velocity of the car at \(t = 2\) seconds is 14 meters
per second. \(\square\)

\subsection{Temperature Change}

The temperature \(T(x)\) in a metal rod \(x\) cm from one end is given by
\(T(x) = 100 - 2x^2\). Find the rate of change of temperature when \(x = 5\)
cm.

\textit{Solution} \: The rate of change of the temperature \(x\) centimeters
from the end, in degrees celsius per centimeter, is given by the function
\(T'(x) = -4x\). Evaluating the function at a distance of 5 centimeters gives
\(T'(5) = -4 \cdot 5 = -20\). Thus, at a distance of 5 centimeters from the end
of the rod, the temperature of the rod is \textit{decreasing} at a rate of 20
degrees celsius per centimeter. \(\square\)

\subsection{Bacterial Growth}

The number of bacteria in a culture at time \(t\) (in hours) is modeled by
\(N(t) = 500e^{0.1t}\). Find the rate at which the bacteria population is
growing at \(t = 3\) hours.

\textit{Solution} \: As with the previous two problems, we can simply
differentiate \(N\) to find the rate of change. The rate of change of \(N\) at
a time \(t\), in bacteria per hour, is given by \(N'(t) = 50e^{0.1t}\). At \(t
= 3\) hours, the rate of change in the number of bacteria is given by
\(T'(3) = 50e^{0.1 \cdot 3} = 50e^{0.3} \approx 67.493\). Thus, three hours
after the beginning of the experiment, the bacteria are growing at a rate of
approximately 67.493 bacteria per hour. \(\square\)

\section{Derivatives}

\subsection{Revenue Maximization}

A company sells \(x\) units of a product at a price \(p(x) = 100 - 2x\) dollars
per unit. What number of units \(x\) maximizes the revenue \(R(x) = x \cdot
p(x)\)?

\textit{Solution} \: Expanding the definition of \(R\) gives \(R(x) = x \cdot
(100 - 2x) = 100x - 2x^2\). We can now see that \(R\) is a parabola which opens
downward. That means that its maximum value will be attained precisely at the
value of \(x\) for which \(R'(x) = 0\). Solving for \(x\):

\[
  \begin{split}
    R'(x) & = 0 \\
    100 - 4x & = 0 \\
    -4x & = -100 \\
    x & = 25
  \end{split}
\]

This means that manufacturing 25 units will maximize the company's profit.
\(\square\)

\subsection{Water Leak}

Water leaks from a tank at a rate modeled by \(r(t) = 10 - 2t\) liters per
hour. At what time \(t\) does the leak stop?

\textit{Solution} \: The leak stops when the rate of the leak reduces to zero.
That means we should solve for \(t\) in \(r(t) = 0\):

\[
  \begin{split}
    r(t) & = 0 \\
    10 - 2t & = 0 \\
    -2t & = -10 \\
    t & = 5
  \end{split}
\]

Thus, the leak stops at \(t = 5\) hours. \(\square\)

\subsection{Falling Object}

A ball is dropped from a height of 100 meters. Its height \(h(t)\) at time
\(t\) (seconds) is given by \(h(t) = 100 - 4.9t^2\). Find the velocity of the
ball when it hits the ground.

\textit{Solution} \: The ball will hit the groud at the time value \(t\)
satisfying \(h(t) = 0\), so let us calculate that first:

\[
  \begin{split}
    h(t) & = 0 \\
    100 - 4.9t^2 & = 0 \\
    -4.9t^2 & = -100 \\
    t^2 & = \frac{1000}{49} \\
    t^2 & = \pm\sqrt{\frac{1000}{49}} \\
    t & = \pm\frac{\sqrt{1000}}{\sqrt{49}} \\
    t & = \pm\frac{10\sqrt{10}}{7}
  \end{split}
\]

We can take either the positive or negative value, but since \(t\) represents a
time value, only the positive one makes sense. We can then plug that value of
\(t\) into the velocity function to get the velocity of the ball at that time.
We then have

\[
  \begin{split}
    h'(\frac{10\sqrt{10}}{7})
    & = -9.8 \cdot \frac{10\sqrt{10}}{7} \\
    & = \frac{-98\sqrt{10}}{7} \\
    & \approx -44.272
  \end{split}
\]

Thus, the precise moment when the ball hits the ground, it has a velocity of
44.272 meters per second in the downward direction. \(\square\)

\subsection{Car Acceleration}

A car's velocity (in m/s) is given by \(v(t) = 4t^2 - 2t + 1\). Find the car's
acceleration at \(t = 3\) seconds.

\textit{Solution} \: The acceleration is given by the derivative of the
velocity. Let \(a(t) = 8t - 2\) represent the acceleration of the car, in
meters per second squared, at a time \(t\). The acceleration of the car at \(t
= 3\) is then given by \(a(3) = 8 \cdot 3 - 2 = 24 - 2 = 22\). Thus, at time
\(t = 3\) seconds, the car's velocity is increasing at a rate of 22 meters per
second, per second. \(\square\)

\section{Area and Volume}

\subsection{Area Between Curves}

Find the area between \(y = x^2\) and \(y = 2x\) for \(0 \leq x \leq 2\).

\textit{Solution} \: Before going any further, let \(f(x) = x^2\) and \(g(x) =
2x\). Picturing these two curves, we know that \(\forall x \in [0, 2]. f(x)
\leq g(x)\). That means that we have to take the total area enclosed by \(g\),
the \(x\)-axis, and the line \(x = 2\) and subtract from it the area enclosed
by \(f\), the \(x\)-axis, and the line \(x = 2\). We can use integrals:

\[
  \begin{split}
    \int_{0}^{2} g(x) dx - \int_{0}^{2} f(x) dx
    & = \int_{0}^{2} 2x dx - \int_{0}^{2} x^2 dx \\
    & = (2^2 - 0^2) - (\frac{1}{3} \cdot 2^3 - \frac{1}{3} \cdot 0^3) \\
    & = (4 - 0) - (\frac{1}{3} \cdot 8 - \frac{1}{3} \cdot 0) \\
    & = 4 - (\frac{8}{3} - 0) \\
    & = 4 - \frac{8}{3} \\
    & = \frac{4}{3}
  \end{split}
\]

Thus, the area between the two curves is \(4/3\) units. \(\square\)

\section{Epsilon-Delta Problems}

\subsection{Limit of a Linear Function}

Prove that

\[
  \lim_{x \rightarrow 2} (3x + 1) = 7.
\]

\textit{Solution} \: Let \(f(x) = 3x + 1\). We aim to show that for any
\(\varepsilon > 0\) such that \(|f(x) - 7| < \varepsilon\), there exists some
\(\delta > 0\) such that \(|x - 2| < \delta\).

\[
  \begin{split}
    |f(x) - 7| & < \varepsilon \\
    |3x + 1 - 7| & < \varepsilon \\
    |3x - 6| & < \varepsilon \\
    3|x - 2| & < \varepsilon \\
    |x - 2| & < \frac{\varepsilon}{3}
  \end{split}
\]

We can then take \(\delta = \frac{\varepsilon}{3}\), since as we just saw, this
quantity satisfies the inequality. Thus, there exists such a \(\delta\) as
required. \(\square\)

\subsection{Limit of a Quadratic Function}

Prove that

\[
  \lim_{x \rightarrow 1} (x^2 + 2x) = 3.
\]

\textit{Solution} \: Let \(f(x) = x^2 + 2x\). We aim to show that for any
\(\epsilon > 0\) there exists some \(\delta > 0\) such that \(|x - 1| < \delta
\Rightarrow |f(x) - 3| < \epsilon\).

Given \(\epsilon > 0\), choose \(\delta = \min\{1, \frac{\epsilon}{3}\}\).

\[
  \begin{aligned}
    |f(x) - 3| & < \epsilon \\
    |x^2 + 2x - 3| & < \epsilon \\
    |x^2 + 3x - x - 3| & < \epsilon \\
    |x(x + 3) - (x + 3)| & < \epsilon \\
    |(x + 3)(x - 1)| & < \epsilon \\
    |x + 3||x - 1| & < \epsilon \\
  \end{aligned}
\]

\end{document}
