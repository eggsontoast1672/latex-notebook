\documentclass{beamer}

\usepackage{mathtools}

\title{Isomorphisms: The Bigger Picture of Math}
\author{Paul Zupan}
\institute{First Year Undergraduate \\ University of Oregon}
\date{Math For All, April 2025}

\newtheorem{remark}{Remark}

\DeclareMathOperator{\id}{id}
\DeclareMathOperator{\ob}{ob}

\DeclarePairedDelimiter{\abs}{\lvert}{\rvert}
\DeclarePairedDelimiter{\inner}{\langle}{\rangle}

\begin{document}

\frame{\titlepage}

\begin{frame}
  \frametitle{About Me}

  \begin{itemize}
    \item First year undergrad at UO studying math and music
    \item Software development background
    \item Aiming for a master's and PHD so I can teach math
  \end{itemize}
\end{frame}

\begin{frame}
  \frametitle{Basic Category Theory}

  \begin{definition}
    A category \(\mathcal{C}\) consists of the following:

    \begin{itemize}
      \item A collection of objects (\(\ob{\mathcal{C}}\))
      \item A collection of morphisms (\(f : A \to B, g : B \to C\))
      \item Associative morphism composition (\(g \circ f\))
      \item Identity morphisms (\(f \circ \id_A = f = \id_B \circ f\))
    \end{itemize}
  \end{definition} \pause

  \begin{examples}
    \begin{itemize}
      \item Sets and functions (\(\mathbf{Set}\))
      \item Groups and homomorphisms (\(\mathbf{Grp}\))
      \item Vector spaces and linear maps (\(K\)-\(\mathbf{Vect}\))
    \end{itemize}
  \end{examples}
\end{frame}

\begin{frame}
  \frametitle{Isomorphisms}

  \begin{definition}
    Let \(f : A \to B\) be a morphism. We say \(f\) is an isomorphism if there
    exists an arrow \(f^{-1} : B \to A\) such that \(f^{-1} \circ f = \id_A\)
    and \(f \circ f^{-1} = \id_B\).
  \end{definition}

  \begin{itemize}
    \item An isomorphism is an invertible arrow
    \item Objects \(A\) and \(B\) contain the same information
    \item Defines an equivalence relation \(A \cong B\)
  \end{itemize}
\end{frame}

\begin{frame}
  \frametitle{Isomorphic Sets}

  \begin{lemma}
    Let \(A\) and \(B\) be sets. \(A\) and \(B\) are isomorphic if they contain
    the same number of elements.
  \end{lemma}

  \begin{itemize}
    \item If \(\abs{A} < \abs{B}\), there is no surjection from \(A\) to \(B\)
    \item If \(\abs{A} > \abs{B}\), there is no injection from \(A\) to \(B\)
  \end{itemize} \pause

  \begin{examples}
    The sets \(A = \{a, b, c\}\) and \(B = \{1, 2, 3\}\) are isomorphic.

    \begin{align*}
      a & \mapsto 1 \\
      b & \mapsto 2 \\
      c & \mapsto 3 \\
    \end{align*}
  \end{examples}
\end{frame}

\begin{frame}
  \frametitle{Isomorphic Groups}

  \begin{lemma}
    Let \(G\) and \(H\) be groups. \(G\) and \(H\) are isomorphic if they have
    the same order and same structure.
  \end{lemma}

  \begin{itemize}
    \item Underlying sets are isomorphic
    \item All elements have the same order
    \item Same Cayley table
  \end{itemize} \pause

  \begin{examples}
    The groups \(Z_6\) and \(Z_2 \times Z_3\) are isomorphic. We refer to them
    as \textbf{the} cyclic group of order 6.

    \begin{equation*}
      \begin{aligned}
        0 & \mapsto (0, 0) \\
        1 & \mapsto (1, 1) \\
        2 & \mapsto (0, 2) \\
      \end{aligned}
      \qquad
      \begin{aligned}
        3 & \mapsto (1, 0) \\
        4 & \mapsto (0, 1) \\
        5 & \mapsto (1, 2) \\
      \end{aligned}
    \end{equation*}
  \end{examples}
\end{frame}

\begin{frame}
  \frametitle{Isomorphic Vector Spaces}

  \begin{lemma}
    Let \(V\) and \(W\) be vector spaces over a field \(K\). \(V\) and \(W\)
    are isomorphic if and only if they have the same dimension.
  \end{lemma}

  \begin{itemize}
    \item This is true because the vector space structure is so strong
    \item If fields do not match, isomorphism does not make sense
  \end{itemize} \pause

  \begin{examples}
    The vector spaces \(\mathbb{R}^3\) and \(\mathbb{P}_2\) are isomorphic. It
    is sufficient to define an invertible map on the bases.

    \begin{align*}
      (1, 0, 0) & \mapsto 1 \\
      (0, 1, 0) & \mapsto x \\
      (0, 0, 1) & \mapsto x^2 \\
    \end{align*}
  \end{examples}
\end{frame}

\begin{frame}
  \frametitle{Other Interesting Isomorphisms}

  Many explicit constructions can be made implicit through isomorphism.

  \begin{examples}
    \begin{itemize}
      \item As rings, \(\mathbb{C} \cong \mathbb{R}[x]/\inner{x^2 + 1}\)
      \item As vector spaces, \(\mathbb{C} \cong \mathbb{R}^2\)
      \item As vector spaces, \(V \times W \cong V \oplus W\)
      \item As groups, \(D_3 \cong S_3\)
      \item Define an equivalence relation \(\sim\) on \(\mathbb{N}^2\) by
        \((a, b) \sim (c, d)\) if and only if \(a + d = b + c\). Then
        \(\mathbb{Z} \cong \mathbb{N}^2/(\sim)\) as sets.
    \end{itemize}
  \end{examples}
\end{frame}

\begin{frame}
  \frametitle{Final Thoughts}

  \begin{itemize}
    \item Implicit is better than explicit
    \item Imprecision without loss of precision
    \item What does ``is'' mean?
    \item Aside: the Yoneda lemma
    \item Recommended reading:
      \begin{enumerate}
        \item The Joy of Abstraction by Eugenia Cheng
        \item Undergraduate Algebra by Serge Lang
        \item Categories for the Working Mathematician by Saunders Mac Lane
      \end{enumerate}
  \end{itemize} \pause

  \begin{center} \Huge
    Thank you for coming!
  \end{center}
\end{frame}

\end{document}
